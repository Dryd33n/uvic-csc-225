\documentclass{article}
\usepackage[legalpaper, portrait, margin=0.5in]{geometry}
\usepackage{amsmath}
\usepackage{amssymb}
\usepackage{enumitem}
\usepackage{listings}
\usepackage{xcolor}

%New colors defined below
\definecolor{codegreen}{rgb}{0,0.6,0}
\definecolor{codegray}{rgb}{0.5,0.5,0.5}
\definecolor{codepurple}{rgb}{0.58,0,0.82}
\definecolor{backcolour}{rgb}{0.95,0.95,0.92}

%Code listing style named "mystyle"
\lstdefinestyle{mystyle}{
  backgroundcolor=\color{backcolour}, commentstyle=\color{codegreen},
  keywordstyle=\color{magenta},
  numberstyle=\tiny\color{codegray},
  stringstyle=\color{codepurple},
  basicstyle=\ttfamily\footnotesize,
  breakatwhitespace=false,         
  breaklines=true,                 
  captionpos=b,                    
  keepspaces=true,                 
  numbers=left,                    
  numbersep=5pt,                  
  showspaces=false,                
  showstringspaces=false,
  showtabs=false,                  
  tabsize=2
}

%"mystyle" code listing set
\lstset{style=mystyle}


\title{CSC 225 Assignment 2}
\author{Dryden Bryson}
\date{\today}

\begin{document}

\maketitle
\newpage
\section*{Question 1:}
\subsection*{a)}
We count the following assignments and comparisons:
\begin{itemize}
    \item 1 Assignment on line 1: $\boxed{p \leftarrow 1}$
    \item For loop, 1 Assignment to initialize $i$: $\boxed{ i \leftarrow 1}$ and 1 comparison to terminate loop
    \item Within the for loop, $n^{2}+1$ times: \begin{itemize}
        \item 1 Comparison to check loop condition $\boxed{i\leftarrow 1 \text{ to } n^{2}+1}$ or $\boxed{i\leq n^{2}+1}$
        \item 1 Assignment to increment $i$, $\boxed{i\leftarrow i+1}$
        \item 1 Assignment in body of loop $\boxed{p\leftarrow p\cdot i}$
    \end{itemize}
\end{itemize}
This gives us, $1A+1A+1C$ comparisons and assigmnets which occur once and $1C+1A+1A$ which occur $n^{2}+1$ times in the for loop. This gives us the following: $$T_{A\&C}(n)=3+\sum_{i=1}^{n^{2}+1}3$$

\subsection*{b)}
We count the following assignments and comparisons:
\begin{itemize}
    \item 1 Assignment on line 1: $\boxed{s\leftarrow 1}$
    \item First for loop, 1 Assignment to initialize $i$: $\boxed{i\leftarrow 0}$ and one comparison upon loop termination
    \item Within first for loop, $n^{2}+1$ times: \begin{itemize}
        \item 1 Comparison to check loop condition, $\boxed{i\leftarrow 1 \text{ to } n^{2}+1}$ or $\boxed{i\leq n^{2}+1}$
        \item 1 Assignment to increment $i$, $\boxed{i\leftarrow i+1}$
        \item Nested for loop, 1 assignment to initialize $j$, $\boxed{j \leftarrow 1}$ and one comparison upon loop termination
        \item Within nested loop $i$ times, \begin{itemize}
            \item 1 Comparison to check loop condition $\boxed{j\leftarrow 1} \text{ to } i$ or $\boxed{j\leq i}$
            \item 1 Assignment to increment $j$, $\boxed{j\leftarrow j+1}$
            \item 1 Assignment within body of loop: $\boxed{s\leftarrow s + i}$ 
        \end{itemize}
    \end{itemize}
\end{itemize}
This gives us, $1A+1A+1C$ assignments and comparisons which occur once and $1C+1A+1A+1C$ within the first loop occuring $n^{2}+1$ times and within that loop $1C+1A+1A$ which occur $i$ times. This gives us the following: $$T_{A\&C}(n)=3+\sum_{i=0}^{n^{2}+1}\bigg[4+\sum_{j=1}^i 3\bigg]$$
or equivalently, avoiding nested summations: $$T_{A\&C}(n)=3+\sum_{i=0}^{n^{2}+1}\bigg[4+3i\bigg]$$

\newpage
\section*{Question 2:}
The following algorithm recursively finds the minumum and the maximum of an array:
\begin{lstlisting}
Algorithm findMinMax(A,n)
    Input: An array A storing atleast 1 real number, length of array denoted by n 
    Output: A pair (min,max) containing the minimum and maximum values of the array

    if n = 1 then          //if array only contains one item
        return (A[0],A[0]) //return only element as min and max
    else
        (min, max) = findMinMax(A,n-1)

        if A[n - 1] < min then
            min = A[n - 1]  
        else if A[n - 1] > max then
            max = A[n - 1]  

        return (min,max)
\end{lstlisting}
For a worst case analysis, we will asume with each recursive iteration finds a new max so the first if statement is false resulting in 2 comparisons. We count the following assignments and comparisons:
\begin{itemize}
    \item Base Case $n=1$
    \begin{itemize}
        \item 1 Comparison for \boxed{n=1}
        \item 1 Assignment for \boxed{\text{return(A[0],A[0])}}
        \item \textbf{Total 2 Assignments and Comparisons}
    \end{itemize}
    \item Recursive case $n>1$
    \begin{itemize}
        \item 1 Comparison for \boxed{n=1}
        \item 2 Assignments and $+T(n-1)$ for \boxed{\text{( min , max ) = findMinMax (A ,n -1)}}
        \item 1 Comparison for \boxed{\text{if A [ n - 1]} < \text{min then}}
        \item 1 Comparison for \boxed{\text{else if A [ n - 1]} > \text{max then}}
        \item 1 Assignment for \boxed{\text{min = A [ n - 1]}} or \boxed{\text{max = A [ n - 1]}} because of if-else flow
        \item 1 Assignment for \boxed{\text{return ( min , max )}}
    \end{itemize}
\end{itemize}

Thus the recurrence equation counting the number of assignments and comparison is as follows: $$T(n)=\begin{cases}
2 &\text{if } n=1\\
T(n-1)+7 &\text{if } n>1
\end{cases}$$
 
\newpage
\section*{Question 3:}
Let us first define the following for simplicity:
\begin{itemize}
    \item $T(n-1)=T(n-2)+2^{n-1}$
    \item $T(n-2)=T(n-3)+2^{n-2}$
    \item $T(n-3)=T(n-4)+2^{n-3}$
\end{itemize}
Then we can proceed with the definition of $T(n)$ and make repeated top down substitutions:
$$\begin{aligned}
    T(n)&=T(n-1)+2^{n}\\
    &=T(n-2)+2^{n-1}+2^{n}\\
    &=T(n-3)+2^{n-2}+2^{n-1}+2^{n}\\
    &=T(n-4)+2^{n-3}+2^{n-2}+2^{n-1}+2^{n}\\
\end{aligned}$$
Thus we can see the pattern that arrises which yeilds the reccurence: $$T(n)=T(0)+2^{1}+2^{2}+2^{3}+\dots+2^{n}$$
When we apply the base case $T(0)=1$ the equation becomes: $$T(n)=1+\sum_{i=1}^n 2^{i}$$
Since $\sum_{i=1}^n 2^{i}$ is a geometric series it can be represented by the closed form $\frac{2(2^{n}-1)}{2-1}$ which simplifies to $2(2^{n}-1)$
Thus we have that: $$T(n)=1+2(2^{n}-1)$$


\newpage
\section*{Question 4:}
\begin{enumerate}[label=\roman*.]
  \item \textbf{Basis}\newline
  We want to show that the statement holds when $n=1$, that is that the given formula we aim to prove yeilds 1 when $n=1$ since $T(1)=1$ is given, when $n=1$: $$\frac{1(1+1)}{2}=\frac{2}{2}=1$$
  Thus the base case holds.
  \item \textbf{Inductive Hypothesis}\newline
  Assume that the frmula holds true for $n>1$ i.e: $$T(n)=\frac{n(n+1)}{2}$$
  \item \textbf{Inductive Step}\newline
  We want to show that the statement holds true for $n+1$ so we have that:
  \begin{table}[htp]
  \centering
  \begin{tabular}{ccll}
    $T(n+1)$ &  $=$ & $T(n)+(n+1)$  &  From the given reccurence relation \\
             &  $=$ & $\frac{n(n+1)}{2}+(n+1)$ & From the Inductive Hypothesis \\
             &  $=$ & $\frac{n(n+1)}{2}+\frac{2(n+1)}{2}$ & \\
             &  $=$ & $\frac{n(n+1)+2(n+1)}{2}$&  \\
             &  $=$ & $\frac{(n+1)(n+2)}{2}$& Factor $(n+1)$ \\
  \end{tabular}
  \end{table}
  The above matches the formula we aim to prove when $n+1$ or that $T(n+1)=\frac{(n+1)(n+2)}{2}$.

  \item \textbf{Conclusion}\newline
  By induction, $T(n)=\frac{n(n+1)}{2}$ holds for all $n\geq 1$
\end{enumerate}


\newpage
\section*{Question 5:}
\textbf{Loop Invariant Proof:}\\
\textit{Loop Invariant:} Throughout the loop it is true that at all times the variable sum holds the sum of the first $k$ elements of the array.

\begin{enumerate}[label=\roman*.]
  \item \textbf{Initialization}\newline
        Before the loop starts $k=1$, and $\text{sum}\leftarrow \text{A}[0]$, which is indeed the sum of the first $k=1$ elements of A thus the loop invariant holds.
  \item \textbf{Maintenance}\newline
        We assume that the invariant holds at the start of an interation, that is that sum holds the sum of the first $k-1$ elements. During the loop the value of the $k$'th element of A is added to sum so the loop invariant holds since sum now stores the sum of the first $k$ elements.
  \item \textbf{Termination}\newline    
        The loop terminantes when $k=n-1$ or once all of the elements of A have been indexed by the loop and added to the sum variable, thus sum holds the sum of the first $k$ elements of the array which at this point is all of the elements in the array, sum is then returned which is exactly the desired output of the algorithm
\end{enumerate}
The loop invariant holds true before, during and after every iterations of the loop. Therefore, the algorithm arraySum(A,n) correctly computes the sum of the elements of the array A. $\square$

\end{document}